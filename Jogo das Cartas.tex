\documentclass[12pt, a4paper]{article}
\usepackage[portuguese]{babel}
\usepackage[margin=0.6in]{geometry}
\usepackage{indentfirst}

\begin{document}

\title{\textbf{Jogo das Cartas}}
\author{}
\date{}
\maketitle

\pagenumbering{gobble}

Número de jogadores: 2

Materiais necessários: 1 baralho (sem os coringas) e 1 dado de 6 faces.

\section*{Preparando o Jogo}
\begin{enumerate}
    \item Role o dado 8 vezes e some os resultados. Este será o número de cartas utilizadas na partida.
    \item Separe em uma pilha o número de cartas (obtido no passo anterior) de um baralho já embaralhado.
    \item Role o dado mais uma vez. Este será o número máximo de cartas que um jogador pode retirar por rodada. O resultado deve ser maior que 1 e menor que 6 (caso o dado caia no 1 ou no 6, role novamente até obter um número válido).
\end{enumerate}

\section*{Durante o Jogo}
Cada jogador deve rolar o dado uma vez. O jogador que tirar o número mais alto começa.

\medskip

Em suas respectivas rodadas, os jogadores devem retirar até $n$ cartas da pilha, onde $n$ é o número sorteado no passo 3 da preparação. Por exemplo, caso o dado tenha caído no 4, cada jogador pode retirar 1, 2, 3 ou 4 cartas da pilha em sua rodada.

\medskip

O jogo acaba quando a última carta for retirada da pilha. Nesse momento, os jogadores devem somar os valores de suas cartas, informados na seção Valor das Cartas.

\section*{Quem Vence o Jogo?}
O jogo possui dois critérios para definição do vencedor:

\begin{itemize}
    \item Quem retirar a última carta da pilha;
    \item Aquele cuja soma das cartas for mais próxima à metade do número de cartas sorteado.
\end{itemize}

Os jogadores devem escolher qual prêmio cada vencedor receberá.

\section*{Valor das Cartas} \label{Valor das Cartas}
As cartas numéricas têm valor correspondente a seu número (por exemplo, cartas de número 2 têm valor 2, cartas número 3 têm valor 3, e assim por diante).

\medskip

Cartas figuradas possuem os seguintes valores: Ás (A) = 1; Valete (J) = 11; Dama (Q) = 12; Rei (K) = 13.

\medskip

Não há diferença de peso entre os naipes, isto é, cartas de mesmo número ou figura de naipes diferentes possuem o mesmo valor.

\end{document}